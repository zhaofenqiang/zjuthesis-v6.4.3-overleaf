\cleardoublepage
\chapternonum{摘\quad 要}
MRI技术的问世与发展使得我们可以无创地采集、观察人类大脑图像并研究大脑的神经系统发育与疾病。在过去几十年中,MR图像处理的技术蓬勃发展,其中最关键的便是基于脑体积和基于脑皮层的数据处理流程(pipeline)。脑体积的数据处理流程中已经出现了很多基于深度学习的算法,提高了它的速度与精度,然而脑皮层的数据处理与分析大多还依赖传统的手工提取特征和机器学习方法,在很多应用上已经满足不了当前大规模神经影像数据处理的需求。因此,为了更加快速、准确地得到脑皮层数据的处理结果,本论文针对当前脑皮层分析pipeline中存在的问题进行深入探究,利用深度学习中的卷积神经网络(Convolutional Neural Network, CNN)极大地加快了数据处理流程,同时还获得了比当前方法高得多的准确度。具体地,本论文重点围绕以下四个方面的工作展开。

(1)基于球面U-Net的皮层分区与属性预测。大脑皮层表面的结构是在manifold空间中具有球形拓扑属性的,其上的每一个顶点之间没有一致的邻域定义,因而无法在皮层表面上进行直接的卷积操作并搭建CNN。为了解决这个问题,在本论文中,我们利用重新采样到球面空间的皮层表面的规则几何结构,提出了一种新型的卷积滤波器——1-ring卷积核。然后基于1-ring卷积核,我们在球面上开发了相应的卷积、池化和转置卷积等操作,从而构建了球面U-Net(Spherical U-Net)结构。随后,我们将球面U-Net应用于大脑皮层数据处理中两个具有挑战性的任务:皮层表面分区和皮层属性发育预测。这两个应用都证明了我们的球面U-Net与当前最好的方法相比,在准确性和计算效率上具有更加优越的性能。

(2)基于三个正交球面U-Net的快速球面配准算法。皮层表面配准是将不同个体和时间点的皮层表面对齐,以建立横向和纵向的皮层对应关系。在本论文中,我们基于CNN开发了一个快速球面配准(Superfast Spherical Surface Registration, S3Reg)算法。S3Reg采用了端到端的无监督学习策略,因此在输入特征和输出相似度上提供了很大的灵活性,同时还显著地降低了配准时间。为了解决极点失真问题,我们设计了三个正交球面U-Net网络结构,利用球面CNN强大的学习能力直接学习球面空间中的速度场,并通过“缩放和平方”层得到拓扑保证的变形场。分别配准成人和婴儿的多模态皮层特征的实验结果表明,我们的S3Reg获得了与当前最好方法相当的性能,但同时将配准时间从1分钟降低到了10秒。	

(3)基于球面神经网络的大脑皮层同时分区与配准。传统的大脑皮层配准与分区是两个独立的任务,但这明显忽略了他们之间的密切联系。为此,我们提出了一个同时进行配准和分区的深层球面神经网络。我们首先使用一个编码器学习这两个任务的共享特征,然后分别为配准和分区训练一个特定的解码器。随后,我们进一步挖掘这两个任务之间的明确联系,加入了分区图相似性损失来加强感兴趣区域边界的一致性,从而为配准任务提供额外的监督信息。反过来,通过将一个带有人工标记的分区图配准到另一个没有人工标记的球面上可以为分区任务提供额外的训练数据。实验结果表明,我们的方法相比单独训练的分区和配准模型获得了巨大的提升,并且能够使用很少的标记数据来训练高质量的分区和配准模型。

(4)基于球面U-Net的多中心皮层属性数据结合方法。直接结合多中心的神经影像数据,例如大脑皮层厚度,进行联合分析将不可避免地引入MRI扫描仪的差异。为了解决这个问题,我们将球面U-Net和周期一致性对抗网络结合起来,提出了一个球面皮层表面的周期一致性对抗网络——S2SGAN。我们将扫描仪$X$到扫描仪$Y$的数据结合建模为球面之间的数据映射任务。因此,我们要学习一个映射$G_X:X\rightarrow Y$,使得$G_X(X)$的数据分布与$Y$无法区分。其次,我们还要保留个体差异,便利用逆映射$G_Y:Y\rightarrow X$和周期一致性损失让$G_Y(G_X(X))\approx X$(反之亦然)。我们在合成的和真实的脑皮层数据上的定量和定性结果都表明,与当前最好的方法相比,我们的方法在消除不必要的扫描仪效应和保留个体差异方面都获得了更好的结果。



\textbf{关键词}:球面网络;神经网络;卷积神经网络;皮层分区;皮层配准;无监督学习;大脑发育模型;数据结合






\cleardoublepage
\chapternonum{Abstract}
The advance of MRI technology makes it possible to acquire and observe human brain images, and thus study the neurological development and diseases in a non-invasive way. Over the past few decades, MR image processing techniques have been developed and matured, in which the most critical steps are brain volume-based analysis and cortical surface-based analysis. Recently, many deep learning-based algorithms have been developed to improve the speed and accuracy of volume-based analysis, but surface-based analysis still relies on the traditional hand crafted features and machine learning algorithms, which can no longer meet the requirements of current large-scale neuroimaging studies. Therefore, to obtain faster and more accurate results of surface-based analysis, in this thesis, we present an comprehensive study on the use Convolutional Neural Networks (CNNs) for analyzing cortical surface data. Specifically, we focus on the following four subjects.

(1) Spherical U-Net-based cortical surface parcellation and development prediction. As the structure of cortical surface has a spherical topology in a manifold space, there is no consistent neighborhood definition and thus no straightforward convolution/transposed convolution operations for cortical surface data. In this thesis, by leveraging the regular and consistent geometric structure of the resampled spherical cortical surface, we propose a novel convolution filter, called 1-ring filter. Accordingly, we develop corresponding operations for convolution, pooling, and transposed convolution for spherical surface data and thus construct spherical U-Net architecture. We then apply Spherical U-Net to two important cortical surface analysis tasks: parcellation and development prediction. Both applications demonstrate the competitive performance in the accuracy and computational efficiency of our Spherical U-Net, in comparison with the state-of-the-art methods.
 
(2) Superfast Spherical Surface Registration (S3Reg) based on three orthogonal Spherical U-Nets. Cortical surface registration aligns cortical surfaces across individuals and time points to establish cross-sectional and longitudinal cortical correspondences to facilitate neuroimaging studies. In this thesis, we develop the S3Reg framework by leveraging an end-to-end unsupervised learning strategy. It offers great flexibility in the choice of input feature sets and output similarity measures, and meanwhile reduces the registration time significantly. To handle the polar-distortion issue, we construct the three orthogonal Spherical U-Nets architecture to directly learn the velocity fields on the sphere, and then use 6 "scaling and squaring" layers to guarantee topology-preserving deformations. Experiments are performed to align both adult and infant multimodal cortical features. Results demonstrate our S3Reg shows superior performance with state-of-the-art methods, while improving the registration time from 1 min to 10 sec.	

(3) Joint cortical surface registration and parcellation based on deep spherical neural network. Conventionally, cortical surface registration and parcellation are performed independently as two tasks, ignoring the inherent connections of them. To this end, we propose a deep learning framework for joint cortical surface registration and parcellation. Our approach first uses a shared encoder to learn shared features for both tasks. Then we train two task-specific decoders for registration and parcellation, respectively. We further exploit the more explicit connection between them by incorporating the novel parcellation map similarity loss to enforce the region-of-interests' boundary consistency, thereby providing extra supervision for the registration task. Conversely, warping one surface with manual parcellation map to another surface provides a large amount of augmented data for parcellation task. Experiments on a dataset with more than 600 cortical surfaces show that our approach achieves large improvements over separately trained networks and enables training high-quality parcellation and registration models using much fewer labeled data.
 
(4)	Harmonization of cortical property data based on Spherical U-Net. A joint analysis of cortical properties (e.g., cortical thickness) of multi-site neuroimaging data is unavoidably facing the problem of differences in MRI scanners. To address this issue, in this thesis, we combine Spherical U-Net and CycleGAN to construct a surface-to-surface CycleGAN (S2SGAN) for harmonizing cortical thickness maps between different scanners. Specifically, we model the harmonization from scanner $X$ to scanner $Y$ as a surface-to-surface translation task. The first goal of harmonization is to learn a mapping $G_X: X\rightarrow Y$ such that the distribution of thickness maps from $G_X(X)$ is indistinguishable from $Y$. With the second goal of harmonization to preserve individual differences, we utilize the inverse mapping $G_Y: Y\rightarrow X$ and the cycle consistency loss to enforce $G_Y(G_X(X))\approx X$ (and vice versa). Quantitative evaluation on both synthesized and real cortical data demonstrates the superior ability of our method in removing unwanted scanner effects and preserving individual differences simultaneously, compared to the state-of-the-art methods.
 
\textbf{Keywords}: Spherical network; neural network; convolutional neural network; cortical surface parcellation; cortical surface registration; unsupervised learning; brain growth model; harmonization


