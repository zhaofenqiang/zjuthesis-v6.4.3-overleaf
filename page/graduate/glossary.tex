\cleardoublepage
\chapternonum{术语表}

\begin{center}
    \zihao{5}
    \begin{longtable}{>{\raggedleft}p{0.18\textwidth}  p{0.79\textwidth}}
            MRI   & Magnetic Resonance Imaging,磁共振成像技术或磁共振图像\\
            sMRI & Structural Magnetic Resonance Imaging,结构磁共振成像 \\
            fMRI & Functional Magnetic Resonance Imaging,功能磁共振成像 \\
            tfMRI & Task Functional Magnetic Resonance Imaging,特定任务的功能磁共振成像 \\
            rfMRI & Resting-state Functional Magnetic Resonance Imaging,静息态功能磁共振成像 \\
            DWI  & Diffusion-weighted MRI,扩散磁共振成像 \\
            T1w图像  & T1-weighted (T1w) MR image,T1加权MR图像 \\
            T2w图像  & T2-weighted (T2w) MR image,T2加权MR图像  \\
            CNN     & Convolutional Neural Network,卷积神经网络 \\
            1-ring卷积核 & 1-ring filter,我们设计的球面上的卷积滤波器,详见\ref{sec:1-ring卷积核} \\
            球面U-Net & Spherical U-Net,我们开发的一种球面CNN,用于脑皮层分区、配准等任务,详见\ref{sec:球面U-Net} \\
            Parcellation & 将大脑分为若干个小的感兴趣区域,简称分区,详见\ref{sec:大脑皮层分区实验} \\
            Registration & 图像配准或大脑皮层表面配准,详见\ref{sec:基于三个正交球面U-Net的快速球面配准算法}介绍 \\
            Harmonization & 多中心数据结合,详见\ref{sec:基于球面U-Net的多中心皮层属性数据结合方法} \\
            CycleGAN & Cycle-consistent Generative Adversarial Network,循环一致性生成对抗网络 \\
            S3Reg & Superfast Spherical Surface Registration,我们开发的超快球面配准算法,见\ref{sec:基于三个正交球面U-Net的配准框架}中的详细介绍 \\
            FreeSurfer & 一个流行的神经影像数据处理工具包,见\ref{sec:MR图像预处理}中的相关介绍 \\
            Spherical Demons & 球面Demons算法,一个流行的球面皮层配准算法,见\ref{sec:大脑皮层表面配准}\\
            MSM & Multimodal Surface Matching,另一个流行的球面皮层配准算法,见\ref{sec:大脑皮层表面配准} \\
            Manifold & 流形,本论文中主要讨论一种具体的流形空间中的实例,即大脑皮层表面 \\
            Diffeomorphism & 微分同胚,一种微分流形之间的可逆光滑映射,本论文中主要指光滑的球面变形场,详见\ref{sec:保证球面拓扑属性的变形场}的介绍 \\
            $\phi$  & Spherical deformation field,球面变形场,用于配准球面脑皮层数据,见\ref{sec:基于三个正交球面U-Net的配准框架} \\
            $\overrightarrow{u}$ &  Stationary velocity field,球面上的静态速度场 \\
            $S^2$ & Sphere,一个普通的球面 \\
            $F$ & Fixed image/surface,配准中的固定图像/皮层表面,用来当作配准的模板 \\
            Atlas & 图谱,又称为模板,或者参照物,常与$F$混用 \\
            $M$ & Moving image/surface,需要进行配准的移动图像/皮层表面 \\
            $M\circ \phi$ & Moved image/surface,$M$在配准后得到的图像/皮层表面 \\
            ${\{v_n\}}_{n=1}^N$ & 有$N$个顶点的正二十面体离散化球面 \\
            $\mathcal{L}$ & Loss,损失函数 \\
            $X$、$Y$ & 代表采集数据的扫描仪(scanner)、中心(center)或站点(site),详见\ref{sec:基于球面U-Net的多中心皮层属性数据结合方法} \\
            curv      &    Mean curvature,平均曲率,大脑皮层的一个几何形态学特征,详见\ref{sec:大脑皮层分区实验实验设计} \\
            sulc      & Average convexity,平均凸度,大脑皮层的一个几何形态学特征,详见\ref{sec:大脑皮层分区实验实验设计} \\
            thickness & Cortical thickness,大脑皮层厚度  \\
            myelin   & Myelin content,髓鞘含量,大脑的一个功能特征,见\ref{sec:S3Reg的数据和预处理}的介绍 \\
            FGD     & Functional Gradient Density,功能MRI图像导出的功能连接梯度密度特征,在\ref{sec:S3Reg的数据和预处理}中介绍到 \\
            ROI    &  Region of Interest,感兴趣区域 \\
            vertex-wise &  逐个顶点的操作 \\
            ROI-wise &  逐个ROI的操作 \\
            group-wise & 基于一组数据的操作,常用于基于一组数据的配准 \\
            重心插值法 & Barycentric interpolation,一种常用的球面插值算法 \\
            SGD  & Stochastic Gradient Descent,随机梯度下降法,一种流行的训练CNN的方法 \\
            epoch & 训练迭代次数,使用整个训练集完整地训练一次CNN叫作一个epoch \\
            batch size & 批处理量,SGD算法中每一次更新网络参数所使用的数据量   \\
            MSD  & Mean Square Distance,均方差 \\
            MAE & Mean AbsoluteError,平均绝对误差 \\
            PCC  & Pearson Correlation Coefficient,皮尔逊相关系数 \\
            PSNR & Peak Signal-to-Noise Ratio,峰值信噪比 \\
            Dropout & CNN中一种常用的接在特征提取层后的操作 \\
            PCA & Principal Component Analysis, 主成分分析法 \\
    \end{longtable}
\end{center}
