\cleardoublepage
\chapternonum{致\quad 谢}
时光荏苒,转眼间,在求是园的求学生活即将划上句号。回首博士生涯,这五年经历了太多事,遇到了非常多优秀的人,我也从中学习成长了很多。在此,向帮助过我的老师、同学、朋友和家人,致以最诚挚的感谢。

我要感谢我的导师夏顺仁教授,是他带我开启了科研之旅。夏老师从一开始就教导我每一步都要走的稳扎稳打,要潜下心来认真做研究。这是我过去五年及未来几十年的科研路上,一直要遵循的教条。夏老师兢兢业业的工作作风,治学严谨的科研态度,都为我树立了很好的学习榜样。夏老师对我们的指导很认真,每次都认真地帮我们修改论文,从夏老师身上,我学习到了科研的方法和严谨的论文写作方式。夏老师是殷切关怀学生发展,不遗余力为学生铺路搭桥的长者。尤记得,最初我想要做帕金森病的研究时,夏老师带我去浙二找数据,与浙二的老师同学们交流学习等等。还有,我非常感激夏老师为我引荐北卡罗来纳大学教堂山分校的李刚老师,非常大度地支持我出国进行交流学习。在此,我由衷地感谢夏老师这五年来给予我的指导,帮助,支持和鼓励!

我要感谢北卡罗来纳大学教堂山分校的李刚教授,在美国IDEA实验室两年多的时间里,给我的科研工作提出宝贵的建议和指导。李刚老师是做婴儿脑发育的专家,给了我很多具体的指导,并非常认真的帮我修改英文论文,对我发表顶刊顶会提供了非常重要的帮助。我从他那里学到很多做学术研究的技能,包括发现新颖的有意义的课题方向,解决问题的思路和规范的英文写作能力。感谢李老师给予的学习机会和多次参加国际学术会议的机会,带我进入更为广阔的学术天地,激发了我继续在学术路上走下去的欲望和信心。

我要感谢浙江大学第二附属医院的放射科主任张敏鸣教授,***peiyu?,为我。。。。。

我要感谢浙江大学 508 实验室的师兄师姐师弟们:王媛媛、张欢、段丁娜、邱陈辉。学术上,师兄师姐曾给我很多帮助;刚来到美国时,段丁娜师姐给予了我很多生活上与婴幼儿神经影像方面的帮助;