\cleardoublepage
\chapternonum{致\quad 谢}
五年的博士生涯即将画上句号,回首往事,在此向帮助过我的老师、同学、朋友和家人们致以最诚挚的感谢。

首先,我要感谢我的导师夏顺仁教授。夏老师在我一年级刚进入实验室的时候便给我指明了研究方向,让我学习研究最前沿的深度学习技术。这使得我在学术道路上少走了很多弯路,为我后面的科研之路打下了坚实的基础。另外,夏老师严谨的教学风格、求是的科研精神也给刚进入科研之路的我做了很好的榜样。进入二年级后,夏老师便积极帮我与浙二医院联系,获取医学影像数据,合作开展深度学习在神经影像中的应用工作。在我偶尔放松怠惰时,夏老师也会对我进行严肃地批评再加上循循善诱地劝导。三年级时,在夏老师无私的帮助下,我申请到了国家留学基金委的资助,到美国北卡罗来纳大学教堂山分校李刚老师课题组联合培养了两年半。这段时间里,夏老师对我的事情也一直很关心,很多学校学院内部的小事,夏老师都会帮我一一解决,让我安心地在美国顺利连续地做出了很多科研成果。可以说没有夏老师的指导与帮助,我就不可能在5年内顺利毕业!我会将夏老师对我的培养与帮助铭记于心,将严谨的科研态度与求是的科研精神一直传递下去!

其次,我要感谢上海科技大学的沈定刚教授和北卡罗来纳大学教堂山分校的李刚教授,在UNC IDEA Lab两年多的时间里,给我的科研工作提出的宝贵建议和指导。沈老师是领域内顶级的学者,他在组会上提出的很多建议都非常切中要害,帮助我们更好地做学术研究,走上更大的学术平台,非常令人尊重和钦佩。李老师是我在联合培养期间的国外导师,亲自带领我走进了最前沿的婴幼儿大脑发育研究的领域,让我直接站在了众多科研工作者的肩膀上,给我安排了最先进最具前瞻性的课题,即图网络在大脑皮层上的应用。在李老师的指导下,我阅读了很多相关文献,掌握了很多流行的基础工具的使用,随后慢慢摸索出了自己的科研道路,成功发表了第一篇球面网络的文章后,后面的许多工作便水到渠成了。李老师对待工作极端严谨认真的态度也让我受益良多。平时生活中,李老师对我们也非常关心,疫情前我们经常会聚餐聊天,疫情后每次组会也会关心我们的生理心理状态。

我还要感谢北卡教堂山的吴正旺博士。正旺师兄在我刚到美国时手把手地教会了我很多神经影像工具包的使用,在我没有车时还每天送我回家,后面在科研上也经常为我答疑解惑,帮我预处理了很多婴幼儿的大脑皮层数据,让我能够一直专心于自己的工作而不用考虑太多数据的问题。

我要感谢浙江大学第二附属医院的张敏鸣教授,黄沛钰博士,耶尔凡博士,周涛,王淑玥,为我初学神经影像,研究帕金森病与阿尔茨海默症时提供的医学知识与实验设备方面的帮助。

我要感谢508实验室的师兄师姐们:王媛媛,张欢,段丁娜,邱陈辉。学术上,师兄师姐在我迷茫时曾帮我想了很多可能的方向,让我自己选择;生活中,师兄师姐们也经常带我一起玩,爬老和山,逛西湖,在南门外赏樱花等等,都是非常美好的回忆!刚来到美国时,段丁娜师姐也给予了我很多生活上的帮助。

我要感谢北卡教堂山李刚老师实验室的胡丹老师,张鑫老师,陈赠思老师,王凡师姐,夏菁师姐,陈良骏师兄,王亚师姐,钟涛,裴羽尘,黄影,廖橹帆,娄惊蛟,他们组成了非常可靠的团队,让我感受到了合作的力量。我还要感谢UNC IDEA Lab的其他几位老师:王利教授,刘明霞教授,Pew-Thian Yap教授,他们在组会上对我的科研工作都提过非常宝贵的建议。我也要感谢在UNC IDEA Lab认识的其他好友们,项磊,聂洞,范敬凡,周涛,李春,王帅,魏冬铭,宣锴,孙琨,张慧凤,高鲲,孙悦,周镇,陈羽娜,宁振源,姚东任,刘云碧,吴烨,杨二昆,马磊,以及浙大的桂再援,周雪飞,刘妮娜,刘佳琦,任其康,还有UNC药学院的王萌霖,张予,他们都是很好的玩伴,为平日的生活增添了很多快乐。

我还想感谢我的偶像初音未来,是她让我鼓起勇气学会了日语,让我感受到了各种不同风格的音乐,在我无聊的时候给我的生活增加了很多乐趣。

最后,我想感谢我的家人,爸爸赵迎兵,妈妈臧书玉,大姐赵丹,二姐赵甜甜,是他们无条件的爱与支持让我一步一步走到了今天,还有我的外甥女仲怡瑾,她的天真可爱给我带来了很多欢乐。还要感谢一直努力与坚持的自己,即将毕业,希望能保持好奇与探索的初心,成长为更好的自己!

\rightline{赵奋强}
\rightline{2021年6月于求是园}
