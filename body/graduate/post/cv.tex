\cleardoublepage
\chapternonum{攻读博士学位期间发表的学术论文}


\noindent [1] Spherical Deformable U-Net: Application to Cortical Surface Parcellation and Development Prediction[J].(SCI,\textbf{一作},IF:6.685,中科院SCI期刊分区:Top期刊,医学1区,工程技术1区,计算机1区,生物医学1区,成像科学与照相技术1区,核医学1区)

\noindent [2] S3Reg: Superfast Spherical Surface Registration Based on Deep Learning[J].(SCI,\textbf{一作},IF:6.685,中科院SCI期刊分区:Top期刊,医学1区,工程技术1区,计算机1区,生物医学1区,成像科学与照相技术1区,核医学1区)

\noindent [3] Spherical U-Net on Cortical Surfaces: Methods and Applications[C]. 2019.(EI,\textbf{一作},医学图像分析领域顶级会议)

\noindent [4] Harmonization of Infant Cortical Thickness Using Surface-to-Surface Cycle-Consistent Adversarial Networks[C].(EI,\textbf{一作},医学图像分析领域顶级会议,口头报告)

\noindent [5] Spherical U-Net for Infant Cortical Surface Parcellation[C].(EI,\textbf{一作},医学图像分析领域顶级会议,口头报告)

\noindent [6] Unsupervised Learning for Spherical Surface Registration[C].(EI,\textbf{一作})

\noindent [7] DIKA-Nets: Domain-invariant Knowledge-guided Attention Networks for Brain Skull Stripping of Early Developing Macaques[J].(SCI,二作,IF:5.902,中科院SCI期刊分区:Top期刊,医学1区,神经成像1区) 

\noindent [8] Intrinsic Patch-Based Cortical Anatomical Parcellation using Graph Convolutional Neural Network on Surface Manifold[C].(EI,二作,医学图像分析领域顶级会议)

\noindent [9] Deep Learning Enables Structured Illumination Microscopy with Low Light Levels and Enhanced Speed[J].(SCI,三作,IF:12.121,中科院SCI期刊分区:Top期刊,综合性期刊1区)

\noindent [10] Joint Image Quality Assessment and Brain Extraction of Fetal MRI Using Deep Learning[C].(EI,三作,医学图像分析领域顶级会议)

\noindent [11] Domain-Invariant Prior Knowledge Guided Attention Networks for Robust Skull Stripping of Developing Macaque Brains[C].(EI,三作,医学图像分析领域顶级会议)

\noindent [12] Anatomy-Guided Convolutional Neural Network for Motion Correction in Fetal Brain MRI[C].(EI,三作)

\noindent [13] Multi-branch Deformable Convolutional Neural Network with Label Distribution Learning for Fetal Brain Age Prediction[C].(EI,三作,医学图像分析领域顶级会议)

\noindent [14] 基于非下采样剪切波变换的PET/SPECT和MR图像融合[J].(二作,中文核心期刊)

\noindent [15] 基于卷积稀疏表示的鲁棒性PET和CT图像融合方法[J].(二作,中文核心期刊)